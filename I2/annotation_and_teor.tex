\begin{center}
	\vspace{0.5cm}{\parbox{16cm}{\small{\centering{\textbf{Аннотация}\\
					\hspace{0.6cm} В этом отчёте изложены результаты выполнения лабораторной работы «Изучение электронно-колебательных спектров поглощения двухатомных молекул на примере молекулы I$_2$». Приводится	краткая теория молекулярных спектров, рассматриваются особенности спектра поглощения молекулярного йода в диапазоне температур $T = 310\div350$ К, приводится описание экспериментальной установки и методики регистрации спектров при различных температурах. 
					Исследуется электронно-колебательно спектр поглощения паров иода. Спектр поглощения, исследуемый в работе, соответствует электронному переходу $^1\Sigma^+_g$ и лежит в области длин волн 
					% TODO: точно в этом диапазоне?
					$490 \leq \lambda \leq 650$ нм.
				}}}}
\end{center}

\textbf{\emph{Цель работы:}} изучение структуры электронно-колебательно спектра поглощения двухатомных молекул, определение из спектроскопических данных основных молекулярных постоянных.
\section{Введение}
Оптическая молекулярная спектроскопия является сегодня одним из основных физических методов исследования строения веществ и их взаимодействия. Одним из классических применений данного метода является определение энергетических уровней молекул и связанных с ними молекулярных постоянных. Полученные величины находят применение в расчетах термодинамических функций и констант равновесия химических реакций.

В данной работе по результатам измерения положения кантов полос электронно-колебательно спектра поглощения молекулы и их относительной интенсивности определяется энергия электронного возбуждения состояния $^3\Pi^+_{0u}$ молекулы йода $T'_e$, колебательная частота $\omega_e'$ и коэффициент ангармоничности $x_e'$ для возбуждённого электронного состояния, энергия диссоциации в основном $D''_0$ и возбуждённом $D'_0$ электронных состояниях, равновесное межъядерное расстояние $r_e'$, строится потенциальная кривая состояния $^3\Pi^+_{0u}$.

\section{Теоретическое введение}
При соединении атомов в молекулы их электронные оболочки объединяются.
Электромагнитные поля, возникающие в процессе образования молекулы при
сближении электронных оболочек атомов, уже не являются сферически
симметричными, как в атоме. Появление новых степеней свободы движения
частиц молекулы отражается на структуре её энергетических уровней и,
следовательно, на структуре молекулярного спектра. Наличие в молекуле двух
и более положительно заряженных ядер существенно усложняет рассмотрение
поведения системы заряженных частиц. Если в атоме с помощью квантовой
механики рассматривается распределение вероятности нахождения электронов
в поле только одного ядра, то в случае молекулы необходимо рассматривать
как распределение вероятности нахождения электронов в поле двух и более
ядер, так и вероятность нахождения ядер в пространстве относительно
заданной системы координат.

Из всех свойств атомов и молекул наиболее важно знание их внутренней
энергии $E$. Фундаментальным уравнением, связывающим энергию системы с её
волновой функцией $\Psi$, является стационарное уравнение Шредингера:
\begin{equation}
\label{shr}
\hat H\Psi=E\Psi,
\end{equation}
где $\hat H$ --- оператор полной энергии (гамильтониан). Для того, чтобы
теоретически определить возможные стационарные энергетические состояния
системы частиц (атома, молекулы, иона) и затем по ним рассчитать спектры
или термодинамические функции, необходимо составить для системы оператор
Гамильтона $\hat H$ и решить уравнение \eqref{shr}. Однако точно в аналитическом виде
уравнение Шредингера решается только для простейших модельных систем,
например, гармонического осциллятора, жёсткого ротатора и некоторых
других. Для молекулы уравнение Шредингера настолько усложняется, что его
точное аналитическое решение возможно только для простейшей двухатомной
молекулы – иона при фиксированном положении ядер. Однако бурное
развитие вычислительной техники и создание всё более совершенных и
мощных ЭВМ в последнее время дало новый импульс квантовомеханическим
расчётам молекулярных состояний. В дальнейшем изложении мы будем
ограничиваться в основном наиболее простым случаем двухатомной молекулы,
для которой теория молекулярных спектров наиболее полно разработана.

Для большинства практических задач молекулярной спектроскопии достаточно
точным является приближённое представление полной волновой функции
молекулы в виде произведения
\begin{equation}
\label{psi_var}
\Psi = \Psi_e\Psi_v\Psi_r,
\end{equation}
где индексы $e, v, r$ относятся соответственно к движению электронов,
колебательному движению ядер и к вращательному движению молекулы как
целого. Это позволяет решать уравнение Шредингера \eqref{shr} отдельно для
электронной, колебательной и вращательной волновых функций. В
приближении \eqref{psi_var} полную внутреннюю энергию молекулы можно представить в виде суммы
\begin{equation}
\label{energy}E = E_e+ E_v+E_r,
\end{equation}
где $E_e$ - энергия электронной оболочки молекулы, $E_v$ - энергия колебаний ядер молекулы, $E_r$ - энергия вращения молекулы. Разделение полной волновой функции по типу \eqref{psi_var}, а следовательно, и разделению энергии молекулы на сумму энергий отдельных видов движения возможно при
условии
\begin{equation}
E_e \gg E_v \gg E_r
\end{equation}
что в большинстве случаев выполняется.

Для приближённого описания реальных потенциальных кривых двухатомных молекул наиболее часто используется функция Морзе:
\begin{equation}
\label{eq:morse}
U(r-r_e) = D_e\left[1-e^{-\beta(r-r_e)}\right]^2,
\end{equation}
где $D_e$ --- энергия диссоциации молекулы, отсчитываемой от минимума потенциальной кривой, $\beta = \omega_e \sqrt{\cfrac{2\pi^2\mu c}{D_e h}}$, $\mu$ --- приведенная масса молекулы. Такой потенциальной кривой соответствуют квантованные значения колебательной энергии ангармонического осциллятора
\begin{equation}
\label{eq:levels}
E_v=hc\left[\omega_e\left(v+\frac{1}{2}\right)-\omega_e x_e\left(v+\frac{1}{2}\right)^2\right],
\end{equation}
где $x_e$ --- коэффициент ангармоничности. Выражение \eqref{eq:levels} достаточно точно описывает систему уровней колебательной энергии двухатомных молекул с малой ангармоничностью, таких как, например, молекулы галогенов.

Энергия диссоциации , входящая в уравнение \eqref{eq:morse} и соответствующая
расстоянию от минимума потенциальной кривой до диссоционного предела, чисто теоретическая величина, т.к. молекула не может находиться в
состоянии с колебательной энергией ниже, чем энергия основного состояния с
$v = 0$, значение которой равно
\begin{equation}
E_0 = hc\left[0.5\omega_e-0.25\omega_e x_e\right]
\end{equation}
Тогда
\begin{equation}
D_e = D_0 + E_0
\end{equation}

Спектры поглощения и испускания, возникающие при переходе молекулы из
одного электронного состояния в другое, получили название электронных, хотя
правильнее говорить об электронно-колебательно-вращательном переходе, т.к.
одновременно с электронным изменяется колебательное и вращательное
состояния молекулы. Систему всех
возможных энергетических состояний двухатомной молекулы можно
приближённо представить уравнением

\begin{equation}
E = E_e+E_v+E_r
\end{equation}

$$
E_e + hc\left[\omega_e(v+1/2)-\omega_e x_e(v+1/2)^2\right]
$$

Колебательная структура уровней энергии молекул наиболее полно
проявляется в электронно-колебатлеьно-вращательных полосатых спектрах, в
то время как систему вращательных уровней удобнее изучать по данным ИК
микроволновых колебательно-вращательных и чисто вращательных спектров.
Можно для целей исследования колебательной
структуры пренебречь в уравнении  вращательным термом по сравнению с
электронным и колебательным. Тогда положение полос даётся выражением
\begin{equation}
\nu = \nu_{\text{эл}} + \left[\omega_e'(v'+1/2)-\omega_e' x_e'(v'+1/2)^2\right]-\left[\omega_e''(v''+1/2)-\omega_e'' x_e''(v''+1/2)^2\right]
\end{equation}


Все наблюдаемые в спектре полосы можно разбить на две
группы серий, или прогрессий. Начиная с данного нижнего (или верхнего)
колебательного уровня, возможны серии переходов на все колебательные
уровни верхнего (или нижнего) электронного состояния. Эти серии называются
соответственно v' и v'' --прогрессиями или поперечными и продольными
сериями Деландра. Последнее название связано с тем, что обычно для
проведения анализа колебательной структуры волновые числа наблюдаемых
переходов представляются в виде так называемой таблицы Деландра, в
вертикальных (поперечных) столбцах которой располагаются
последовательные v'-прогрессии, а в горизонтальных (продольных) строках -
v''-прогрессии. Диагональные строки таблицы Деландра соответствуют
группам переходов с фиксированным значением $\Delta v = v' - v''$, которые образуют в
спектрах группы близко расположенных полос или секвенции.
Распределение интенсивности в прогрессиях полос электронно-колебательного
спектра может быть различным в зависимости от взаимного расположения и
вида потенциальных кривых нижнего и верхнего электронных состояний и
объясняется на основе принципа Франка-Кондона, который формулируется
следующим образом. При переходе молекулы из одного электронного
состояния в другое перераспределение электронной плотности происходит так
быстро, что расстояние между более тяжёлыми ядрами не успевает измениться.
Отсюда следует, что на диаграмме кривых потенциальной энергии
квантовые переходы между различными электронными состояниями
двухатомной молекулы должны изображаться вертикальными стрелками,
которые отвечают условию неизменности межъядерного расстояния в процессе
перехода.



