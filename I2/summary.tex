\section{Заключение}
В данной работе рассматривался спектр поглощения молекулы йода. Были получены молекулярные константы, построен потенциал Морзе для возбужденного состояния молекулы $\text{I}_2$.

% TODO: точно у нас так же?
Оказалось, что самый близкий к табличному результат дал метод линейной экстраполяции. 

Анализ спектров при разных ширинах щели показывает, что с уменьшением размера щели увеличивается разрешающая способность прибора. Электронно-колебательную структуру спектра мы начинаем разрешать при ширине щели около 0.5 нм.

Анализ спектра широкого диапазона показывает, что есть 2 характерные области $v''$--прогрессий, а при уменьшении возбуждающей длины волны молекула начинает диссоциировать.